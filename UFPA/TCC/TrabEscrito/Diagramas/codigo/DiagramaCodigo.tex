


%%%%%%%%%%%%%%%%%%%%%%%%%%%%%%%%%%%%%%%%%%%%%%%%%%%%%%%%%%%%%%%
%
% Welcome to Overleaf --- just edit your LaTeX on the left,
% and we'll compile it for you on the right. If you give
% someone the link to this page, they can edit at the same
% time. See the help menu above for more info. Enjoy!
%
% Note: you can export the pdf to see the result at full
% resolution.
%
%%%%%%%%%%%%%%%%%%%%%%%%%%%%%%%%%%%%%%%%%%%%%%%%%%%%%%%%%%%%%%%
\documentclass{article}
\usepackage{tikz}
\usepackage[utf8]{inputenc}
\usepackage{amsmath}
\usetikzlibrary{trees}
%%%<
\usepackage{verbatim}
\usepackage[active,tightpage]{preview}
\PreviewEnvironment{tikzpicture}
\setlength\PreviewBorder{10pt}%
%%%>

\usetikzlibrary{arrows,shapes,positioning,shadows,trees}

\tikzset{
  basic/.style  = {draw,text width=5cm, drop shadow, ellipse,anchor=west},
}


\begin{document}
% Define block styles

\tikzstyle{block1} = [basic, rectangle, draw, fill=blue!60, 
    text width=6em, text centered, rounded corners]
\tikzstyle{block2} = [basic, rectangle, draw, fill=pink!60, 
    text width=6em, text centered, rounded corners]
\tikzstyle{block3} = [basic, rectangle, draw, fill=gray!60, 
    text width=6em, text centered, rounded corners]
\tikzstyle{block4} = [basic, rectangle, draw, fill=orange!60, 
    text width=6em, text centered, rounded corners]
\tikzstyle{block5} = [basic, rectangle, draw, fill=red!60, 
    text width=6em, text centered, rounded corners]
\tikzstyle{block6} = [basic, rectangle, draw, fill=green!60, 
    text width=6em, text centered, rounded corners]
\tikzstyle{block7} = [basic, rectangle, draw, fill=purple!60, 
    text width=6.5em, text centered, rounded corners]    

\tikzstyle{line} = [draw, -latex']
\tikzstyle{cloud} = [basic, rounded corners=6pt, thin,align=center, fill=green!60, text width=5em, node distance=5cm] 

\tikzstyle{selected}=[draw=red,fill=red!30]
\tikzstyle{optional}=[dashed,fill=gray!50]
\begin{tikzpicture}[node distance = 5cm, auto]

% Nós do fluxograma
\node [cloud] (init) {Definir as regiões de interesse};

\node [block3, right of=init] (posicao) {Construir a matriz posição com os pixels das regiões de interesse};

\node [block7, right of=posicao] (reamostragem) {Reamostragem dos pixels das regiões de interesse};

\node [block2, below of=init] (filtros) {Criar um banco com os filtros};

\node [block1, below of=filtros] (respostaFiltros) {Calcular a resposta dos filros à Luminância ($Y$)};

\node [block1, below of=respostaFiltros] (canais) {Calcular os canais com base no resultado anterior};

\node [block2, below of=canais] (normaliza) {Normaliza os $N_{c}$ canais no intervalo [0,255]};

\node [block2, right of=normaliza] (fdps) {Calcular as FDP's das subregiões dos canais};

\node [block4, right of=fdps] (pesocanal) {Calcuar o peso de cada canal baseado na FDP das regiões};

\node [block4, right of=pesocanal] (pesogeo) {Calcuar o peso da distância geodésica por pixel para todas as subregiões};

\node [block5, above of=pesogeo] (distancia) {Calcuar a menor distância de cada pixel para cada subregião};

\node [block6, above of=distancia] (probabilidade) {Calcuar a probabilidade de cada pixel para cada subregião};

\node [block6, above of=probabilidade] (segmentacao) {Define a que subregião o pixel pertence};



% Arestas do fluxograma

\path [line]	(init) -- (posicao) node [midway, above, sloped] (TextNode) {\color{red}{}} ;

\path [line]	(posicao) -- (fdps) node [midway, above, sloped] (TextNode) {\color{red}{Matriz Posição}} ;

\path [line]	(posicao) -- (reamostragem) node [midway, above, sloped] (TextNode) {\color{red}{Matriz Posição}} ;

\path [line]	(reamostragem) |-++ (3.8,-15) (distancia) node [midway, above, sloped] (TextNode) {\color{red}{Matriz Posição Reamostrada}} ;

\path [line] 	(init) -- (filtros); 

\path [line] 	(filtros) -- (respostaFiltros);

\path [line] 	(respostaFiltros) -- (canais);

\path [line] 	(canais) -- (normaliza) node [midway, above, sloped] (TextNode) {\color{red}{Canais}};

\path [line] 	(normaliza) -- (fdps) node [midway, above, sloped] (TextNode) {\color{red}{}};

\path [line] 	(fdps) -- (pesocanal) node [midway, above, sloped] (TextNode) {\color{red}{FDP's}};

\path [line] 	(pesocanal) -- (pesogeo) node [midway, above, sloped] (TextNode) {\color{red}{PesoCanal}};

\path [line] 	(pesogeo) -- (distancia) node [midway, above, sloped] (TextNode) {\color{red}{PesoGeo}};

\path [line] 	(pesogeo) -- (distancia) node [midway, above, sloped] (TextNode) {\color{red}{PesoGeo}};

\path [line] 	(distancia) -- (probabilidade) node [midway, above, sloped] (TextNode) {\color{red}{MinDist}};

\path [line] 	(probabilidade) -- (segmentacao) node [midway, above, sloped] (TextNode) {\color{red}{Probabilidade}};

\end{tikzpicture}

\end{document}