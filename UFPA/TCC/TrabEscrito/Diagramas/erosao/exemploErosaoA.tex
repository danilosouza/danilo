% Highlighting elements in matrices
% Author: Stefan Kottwitz
\documentclass[varwidth=true]{standalone}
\usepackage{tikz}
\pagenumbering{gobble}
\usepackage{amsmath}
\usetikzlibrary{trees}
%%%<

\usetikzlibrary{fit,arrows,matrix,positioning,shapes.misc,shadows}

\tikzset{%
  highlight/.style={rectangle,rounded corners,fill=red!15,draw,fill opacity=0.5,thick,inner sep=0pt}
}
\newcommand{\tikzmark}[2]{\tikz[overlay,remember picture,baseline=(#1.base)] \node (#1) {#2};}
%
\newcommand{\Highlight}[1][submatrix]{%
    \tikz[overlay,remember picture]{
    \node[highlight,fit=(left.north west) (right.south east)] (#1) {};}
}


\begin{document}
\[
	A = \left[\begin{array}{*6{c}}
    \color{green}{0} & \color{green}{1} & 1 & 0 & 0 & 0 \\
    \color{green}{1} & 1 & 1 & 0 & 0 & 0 \\
    0 & 1 & 1 & 1 & 1 & 0 \\
    0 & 0 & 1 & 1 & 1 & 1 \\
    0 & 1 & 1 & 1 & 1 & 1  \\
    0 & 1 & 1 & 1 & 0 & 1 
  \end{array}\right]
  \qquad
  B  = \left[\begin{array}{*3{c}}
    \tikzmark{left}{0} & \color{red}{1} & 0\\
    \color{red}{1} & \color{red}{1} & \color{green}{1} \\
    0 & \color{green}{1} & \tikzmark{right}{0}
  \end{array}\right]
\]

\tikz[overlay,remember picture] {
  %\draw[->,thick,red,dashed] (first) -- (second); 
  %\draw[->,thick,green,dashed] (third) -- (fourth) 
  %\node[above of=first] {$N$};
  %\node[above of=second] {$N^T$};
}

\end{document}?