%%%%%%%%%%%%%%%%%%%%%%%%%%%%%%%%%%%%%%%%%%%%%%%%%%%%%%%%%%%%%%%
%
% Welcome to Overleaf --- just edit your LaTeX on the left,
% and we'll compile it for you on the right. If you give
% someone the link to this page, they can edit at the same
% time. See the help menu above for more info. Enjoy!
%
% Note: you can export the pdf to see the result at full
% resolution.
%
%%%%%%%%%%%%%%%%%%%%%%%%%%%%%%%%%%%%%%%%%%%%%%%%%%%%%%%%%%%%%%%
\documentclass{article}
\usepackage{tikz}
\usepackage[utf8]{inputenc}
\usetikzlibrary{trees}
%%%<
\usepackage{verbatim}
\usepackage[active,tightpage]{preview}
\PreviewEnvironment{tikzpicture}
\setlength\PreviewBorder{10pt}%
%%%>

\usetikzlibrary{arrows,shapes,positioning,shadows,trees}

\tikzset{
  basic/.style  = {draw,text width=5cm, drop shadow, rectangle,anchor=west},
  type 1/.style   = {basic, rounded corners=6pt, thin,align=center, fill=green!60,
                   text width=9em},                
  type 2/.style = {basic, rounded corners=6pt, thin,align=center, fill=blue!60,
                   text width=10em},
  type 3/.style = {basic, thin, align=center, fill=pink!60, text width=9em, rounded corners=6pt},  
  type 4/.style = {basic, thin, align=center, fill=orange!60, text width=9em, rounded corners=6pt},
  type 5/.style = {basic, thin, align=center, fill=gray!60, text width=9em, rounded corners=6pt},
  type 6/.style = {basic, thin, align=center, fill=red!60, text width=9em, rounded corners=6pt},
}

\begin{document}
\tikzstyle{every node}=[draw=black,thick,anchor=west]
\tikzstyle{selected}=[draw=red,fill=red!30]
\tikzstyle{optional}=[dashed,fill=gray!50]
\begin{tikzpicture}[%
  grow via three points={one child at (0.5,-0.7) and
  two children at (0.5,-0.7) and (0.5,-1.4)},
  edge from parent path={(\tikzparentnode.south) |- (\tikzchildnode.west)}]
  \node [type 1] {segmenta}
    child { node [type 5] {getPixelsPosition}}			
	child { node [type 2] {getChannels}
      	child { node [type 3] {gaborFilter}}  
      	child { node [type 3] {getPixelsDist}}  	
	}
    child [missing] {}				
    child [missing] {}				  	
    child { node [type 4] {getChannelWeight}}
    child { node [type 4] {getGeodesicWeight}}		
    child { node [type 6] {getMinDistance}};
\end{tikzpicture}
\end{document}