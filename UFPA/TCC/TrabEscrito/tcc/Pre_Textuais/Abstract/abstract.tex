%
%%%%%%%%%%%%%%%%%%%%%%%%%%%%%%%%%%%%%%%%%%%%%%%%%%%%%%%%%%%%%%%%%%%%%%%%%%%%%%%%%%%%%%%%%%%%%%%%%%%%%%%%%%%%%%%%%%%%%%%%%%%%%%%%%%%%%%%%%%%%%%%%%%%%%%%
%%%%%%%%%%%%%%%%%%%%%%%%%%%%%%%%%%%%%%%%%%%%%%%%%%%%%%%%%%%%%%%%%%%%%%%%%%%%%%%%%%%%%%%%%%%%%%%%%%%%%%%%%%%%%%%%%%%%%%%%%%%%%%%%%%%%%%%%%%%%%%%%%%%%%%%
% 

\chapter*{\ \ \ \ \ \ \ \ \ \ \ \ \ \ \ \ \ \ Abstract}\thispagestyle{empty}

%----------------------------------------------------------------------------------------------------------------------------------------------------%
%

Image segmentation is about splitting an image into objects and/or regions with characteristics in common, such as color, texture and geometry aiming to ease the analysis of the areas of interest within the given image. The growth of mobile devices over the past years, increasing the amount of multimidia content produced, has open the need for more robust and better quality image segmentation algorithms. Implementing a semi-automatic image segmentation technique that relies on color and distance informations of a pixel to classify it accordingly to user defined regions (i.e, set of pixels discribing in terms of texture and color the same region within an image), this work's proposal is to improve execution time spent to calculate the distante from all the pixels in the image to the pixels inside the regions of interest. By adding one more step in order to reduce the search space of the distance calculation step, so it is no longer necessary to calculate it for all the pixels inside the scribbles. The evaluations were done based on execution time and classification erros.

\textbf{KEYWORDS: semi-automatic imagem segmentation, re-sampling, time reducing, image processing, image segmentation}

%
%%%%%%%%%%%%%%%%%%%%%%%%%%%%%%%%%%%%%%%%%%%%%%%%%%%%%%%%%%%%%%%%%%%%%%%%%%%%%%%%%%%%%%%%%%%%%%%%%%%%%%%%%%%%%%%%%%%%%%%%%%%%%%%%%%%%%%%%%%%%%%%%%%%%%%%
%%%%%%%%%%%%%%%%%%%%%%%%%%%%%%%%%%%%%%%%%%%%%%%%%%%%%%%%%%%%%%%%%%%%%%%%%%%%%%%%%%%%%%%%%%%%%%%%%%%%%%%%%%%%%%%%%%%%%%%%%%%%%%%%%%%%%%%%%%%%%%%%%%%%%%%
%