%	
%%%%%%%%%%%%%%%%%%%%%%%%%%%%%%%%%%%%%%%%%%%%%%%%%%%%%%%%%%%%%%%%%%%%%%%%%%%%%%%%%%%%%%%%%%%%%%%%%%%%%%%%%%%%%%%%%%%%%%%%%%%%%%%%%%%%%%%%%%%%%%%%%%%%%%%
%%%%%%%%%%%%%%%%%%%%%%%%%%%%%%%%%%%%%%%%%%%%%%%%%%%%%%%%%%%%%%%%%%%%%%%%%%%%%%%%%%%%%%%%%%%%%%%%%%%%%%%%%%%%%%%%%%%%%%%%%%%%%%%%%%%%%%%%%%%%%%%%%%%%%%%
%

\chapter{Descri��o da t�cnica estudada e Implementa��o}
\label{cap:descricao_tecnica_imp}

%
%----------------------------------------------------------------------------------------------------------------------------------------------------%
%

\section{Descri��o da t�cnica estudada}

A t�cnica implementada neste trabalho, introduzida em \cite{Protiere2007}, pode ser classificada como semi-autom�tica pois necessita da interven��o do usu�rio para marcar as regi�es de interesse da imagem. Estas regi�es podem ser objeto ou fundo, havendo a possibilidade de se marcar mais de um objeto para segmenta��o, nesse caso a imagem resultante seria a soma das imagens de cada objeto separado.

Essa t�cnica parte da premissa de que as regi�es de interesse a serem definidas s�o bem distintas em termos de cor e textura e utilizando o conjunto de pixels marcados $\Omega_{n}$, sendo $n$ o n�mero de regi�es distintas, � calculada a distribui��o gaussiana mostrando a probabilidade de um pixel \textit{p(x,y)} pertencer a uma determinada regi�o $l$. Com base nessas distribui��es s�o calculados pesos para cada canal da imagem que ser�o explicados mais detalhadamente a seguir.

Em \cite{Protiere2007} o autor utilizou 19 canais para segmenta��o, sendo 3 destes canais a Lumin�ncia ($Y$) e Cromin�ncia ($Cr$ e $Cb$) e os outros 16 s�o o resultado da filtragem do canal de $Y$ por 16 filtros diferentes de Gabor, \cite{Manjunath1996} \textcolor{red}{Procurar mais artigos sobre gabor}. O autor utilizou 4 dire��es ($ \theta = 0, \pi/4, \pi/2$ e $ 3\pi/4$) e 4 frequ�ncias centrais ($\omega = 1/2, 1/4, 1/8 $ e $ 1/16 $) para definir os filtros. A escolha de apenas 4 dire��es se d� em fun��o da simetria, uma vez que o sentido n�o importa, ou seja, $0 = \pi$, $\pi/4 = 5\pi/4$, $\pi/2 = 3\pi/2$ e $3\pi/4 = 7\pi/4$ sendo assim poss�vel descrever texturas em qualquer dire��o.



\section{Implementa��o}


%
%%%%%%%%%%%%%%%%%%%%%%%%%%%%%%%%%%%%%%%%%%%%%%%%%%%%%%%%%%%%%%%%%%%%%%%%%%%%%%%%%%%%%%%%%%%%%%%%%%%%%%%%%%%%%%%%%%%%%%%%%%%%%%%%%%%%%%%%%%%%%%%%%%%%%%%
%%%%%%%%%%%%%%%%%%%%%%%%%%%%%%%%%%%%%%%%%%%%%%%%%%%%%%%%%%%%%%%%%%%%%%%%%%%%%%%%%%%%%%%%%%%%%%%%%%%%%%%%%%%%%%%%%%%%%%%%%%%%%%%%%%%%%%%%%%%%%%%%%%%%%%%
%