\documentclass{beamer}

% Setup appearance:

\usetheme{Darmstadt}
\usefonttheme[onlylarge]{structurebold}
\setbeamerfont*{frametitle}{size=\normalsize,series=\bfseries}
\setbeamertemplate{navigation symbols}{}


% Standard packages

\usepackage[brazil]{babel}
\usepackage[latin1]{inputenc}
\usepackage{times}
\usepackage[T1]{fontenc}
\usepackage{amsmath}% http://ctan.org/pkg/amsmath
%\usepackage[table]{xcolor}
\usepackage{multicol}
\usepackage{textcomp} 

% Setup TikZ
\usepackage{tikz}
\usetikzlibrary{arrows}
\tikzstyle{block}=[draw opacity=0.7,line width=1.4cm]

%diretório das figuras
\graphicspath{../figuras}

\title[Extreme Chock]{%
\textcolor{red}{TITULO A DEFINIR}%
}

\author[Souza]{
	\inst{1}%
     Danilo~Souza\and
     }


\institute[Bel�m]{
  \inst{1}%
  Universidade Federal do Par�
  }
\date[Bel�m 2015]{
  25 de Novembro de 2015
  }

\begin{document}

\begin{frame}
  \titlepage
\end{frame}

\begin{frame}
  \tableofcontents
\end{frame}

\section{Introdu��o}

\begin{frame}{Processamento Digital de Imagens}

\end{frame}

\begin{frame}{Aplica��es}

\end{frame}

\begin{frame}{Segmenta��o de Imagens}

\end{frame}


\section{A t�cnica estudada}

\begin{frame}{Introdu��o � t�cnica}

\end{frame}

\begin{frame}{O algoritmo}

\end{frame}

\begin{frame}

\end{frame}

\section{Metodologia utilizada}

\begin{frame}{Modifica��es realizadas}

\end{frame}

\begin{frame}{Escolha dos par�metros avaliados}

\end{frame}

\begin{frame}{M�todo de avalia��o}

\end{frame}


\section{Resultados}

\begin{frame}{Resultados de tempo}

\end{frame}

\begin{frame}{Resultados de erro}

\end{frame}

\begin{frame}{Imagens resultantes}

\end{frame}

\begin{frame}{Avalia��o dos resultados}

\end{frame}

\section{Considera��es finais}

\begin{frame}{Conclus�o}


\end{frame}



\begin{frame}

	{\Huge Obrigado!}

\end{frame}

\end{document}